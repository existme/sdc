\subsection{ZSH}
Shortcuts:

\begin{bashcode} 
		Ctrl+X u			->	undo
		echo $PWD<TAB>		->  echo /home/pws/zsh/projects/zshguide
		echo !!<TAB>		->  echo echo 3.1.7
		echo ~/.z*<TAB>		->  echo /home/pws/.zcompdump /home/pws/.zlogout /home/pws/.zshenv /home/pws/.zshrc
		print ${ZSH_V<TAB>	->	${ZSH_VERSION}
\end{bashcode}

\subsection{Autoload}
autoload can import function definition header (but not the body) so that the functions are loaded once they are used.

To see the functions that are currently loaded via autoload, you can use:\newline
\ic{autoload -U \| less}

Autoload functions are located in "\ip{/usr/share/zsh/functions}". A common pattern for loading these functions is \ic{autoload -U colors && colors}. The colors script is located in \ip{/usr/share/zsh/functions/Misc}.

\subsection{Colors}
One safe way to use colors is defining them via \ic{tput}.

\begin{bashcode}
	BLACK=$(tput setaf 0)
	RED=$(tput setaf 1)
	GREEN=$(tput setaf 2)
	YELLOW=$(tput setaf 3)
	LIME_YELLOW=$(tput setaf 190)
	POWDER_BLUE=$(tput setaf 153)
	BLUE=$(tput setaf 4)
	MAGENTA=$(tput setaf 5)
	CYAN=$(tput setaf 6)
	WHITE=$(tput setaf 7)
	BRIGHT=$(tput bold)
	NORMAL=$(tput sgr0)
	BLINK=$(tput blink)
	REVERSE=$(tput smso)
	UNDERLINE=$(tput smul)
\end{bashcode}
the other way is by loading colors script.
\ic{autoload -U colors &&  colors} \newline
and then using them in code like:
\ic{echo $bg[red]$fg[white] reza \$reset_color}

Inorder to use these colors you can use the following shortcuts:

\begin{bashcode}
	$fg_bold[color]	
	$fg_no_bold[color]
	$bg[color]
	$bg_bold[color]
	$bg_no_bold[color]
	$boldcolor
	$resetcolor	
\end{bashcode}

Or use it in prompt li	ke:\newline
\begin{bashcode}
	PS1="%{$fg[red]%}%n%{$reset_color%}@%{$fg[blue]%}%m %{$fg[yellow]%}%~ %{$reset_color%}%% "
\end{bashcode}

To use other features like underline, overline, etc. you should do a little more:\newline
\begin{bashcode}
local lc=$'\e[' rc=m	# Standard ANSI terminal escape 

echo $fg_bold[red] reza $fg_no_bold[red]shams "$lc"09";${color[white]}$rc" amiri $reset_color 
\end{bashcode}

\begin{bashcode}[The variables that could be used with color]
	color=(
		# Codes listed in this array are from ECMA-48, Section 8.3.117, p. 61.
		# Those that are commented out are not widely supported or aren't closely
		# enough related to color manipulation, but are included for completeness.
		
		# Attribute codes:
		00 none                 # 20 gothic
		01 bold                 # 21 double-underline
		02 faint                  22 normal
		03 standout               23 no-standout
		04 underline              24 no-underline
		05 blink                  25 no-blink
		# 06 fast-blink           # 26 proportional
		07 reverse                27 no-reverse
		08 conceal                28 no-conceal
		# 09 strikethrough        # 29 no-strikethrough
		
		# Font selection:
		# 10 font-default
		# 11 font-first
		# 12 font-second
		# 13 font-third
		# 14 font-fourth
		# 15 font-fifth
		# 16 font-sixth
		# 17 font-seventh
		# 18 font-eighth
		# 19 font-ninth
		
		# Text color codes:
		30 black                  40 bg-black
		31 red                    41 bg-red
		32 green                  42 bg-green
		33 yellow                 43 bg-yellow
		34 blue                   44 bg-blue
		35 magenta                45 bg-magenta
		36 cyan                   46 bg-cyan
		37 white                  47 bg-white
		# 38 iso-8316-6           # 48 bg-iso-8316-6
		39 default                49 bg-default
		
		# Other codes:
		# 50 no-proportional
		# 51 border-rectangle
		# 52 border-circle
		# 53 overline
		# 54 no-border
		# 55 no-overline
		# 56 through 59 reserved
	)
\end{bashcode}

\subsection{alias}
To remove an alias, the unalias command is used, like so:
\ic{unalias foo}

\subsection{Super user setup}
To have a \textit{login shell session} you should use:
\ic{su -} or \ic{su -l}. This means that the user’s environment is loaded and the working directory is changed to the user’s home directory. 

\subsection{check to see if a function exists}
\begin{bashcode}
	if [ type __promptline \&>/dev/null ]; then
		__promptline
	fi       
\end{bashcode}
