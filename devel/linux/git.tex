\section{Branches}

\ic{git checkout -b <new_branch>}

BRANCHES SHOULD ALWAYS BE IN LOWER CASE

Show all branches:
\ic{git fetch origin}

\ic{git branch -v -a }

\ic{git checkout remotes/origin/dev/adp_online_agree_t46324}

\ic{git branch --set-upstream dev/show_nfr_warnings_58187 origin/dev/show_nfr_warnings_58187}


\ic{git fetch origin}

\ic{git branch -v -a}

\ic{git checkout -b test origin/test}

\section{Aliases}

Adding aliases:
Basically you just need to add lines to \ic{\~/.gitconfig}
\begin{lstlisting}
[alias]
st = status
ci = commit -v
Or you can use the git config alias command:
\end{lstlisting}

\ic{git config --global alias.st status }
\ic{git config --global alias.ci 'commit -v'}

A manual alias like \ic{gitall} could be added to see the current listing of aliases:

\ic{alias gitall="git config --list \| grep alias"}

\section{Reverting}
The easiest way is to:
\ic{git reset --hard}
\ic{git reset --hard origin/master}

Recovering a removed file

\ic{git reset HEAD file}

\ic{git checkout -- file}
\\\\
ignore the last commit reverts everything!\\
\ic{git reset --hard "HEAD^"}
\\\\
ignore the last commit keeps everything\\
\ic{git reset --soft "HEAD^"}
\\\\
shows all branches
\ic{gitk --all}
\\\\
Show the changes changes
\ic{vimdiff agreement_new/new.php agreement/new.php}
\\\\
\ic{git branch --set-upstream dev/show_nfr_warnings_58187 origin/dev/show_nfr_warnings_58187}
\\\\
\ic{git status -s}\\
\ic{git commit -a -m "message"} //will add modified files and commit\\
\ic{git config --global color.ui true}\\
\ic{git log --all}

\section{Stashing}

In Git you can drop your current work state in to a temporary storage area stack and then re-apply it later. The simple case is as follows:

\begin{lstlisting}[language=sh,style=mysh]
git stash
git stash pop
git stash drop stash@{1}
git stash save "My stash message"
git stash list
git stash apply stash@{1}
git stash branch					#Create a branch from stash
git stash show -p					#Show the differences
\end{lstlisting}

\begin{lstlisting}[language=sh,style=mysh]
git add -i
git add -p

git log -p
git log --stat
git log --author=Andy
git log --grep="Something in the message"
git log lib/foo.rb
git log --since=2.months.ago --until=1.day.ago
git log --since=2.months.ago --until=1.day.ago --author=andy -S "something" --all-match

git log master ^origin/master
git log origin/master ^master

git unstage lib/foo.rb

git blame lib/foo.rb

git reflog

git mergetool --tool-help 
git config --global alias.lol "log --pretty=oneline --abbrev-commit --graph --decorate"
git config --global alias.unstage "reset HEAD"


git config --global diff.tool vimdiff
git config --global difftool.prompt false
git config --global alias.d difftool
\end{lstlisting}

\section{Squashing commits with rebase}
Use \ic{git lol} to see the logs and decide how many commits you want to squash. Do a rebase by mergin for example last 4 commits.


\ic{git rebase -i HEAD~4}

\textbf{An easier} and safer way would be:
\ic{git merge --squash branch}
\section{Amending last commit}

To change the last commit message:

\section{How to fix cygwin git unnecessary change detection}
Use \ic{git config --local -e} and set \ic{filemode=false}


\begin{lstlisting}[language=sh,style=mysh]
git commit --amend
git commit --amend -m "New commit message"
\end{lstlisting}

\section{How to see the origin address}
\ic{git remote show origin}

\section{How to use a pre-written commit message}
\ic{git commit -F commit.log}

\section{How to save password when using git (see \href{http://stackoverflow.com/questions/5343068/is-there-a-way-to-skip-password-typing-when-using-https-github}{src})}
\begin{itemize}
	\item Linux:\newline
	\begin{bashcode}
		sudo apt-get install libgnome-keyring-dev
		cd /usr/share/doc/git/contrib/credential/gnome-keyring
		sudo make
		git config --global credential.helper
		/usr/share/doc/git/contrib/credential/gnome-keyring/
		git-credential-gnome-keyring
	\end{bashcode}
	\item Mac:\newline
	\begin{bashcode}
		git config --global credential.helper osxkeychain
	\end{bashcode}
	\item Windows:\newline
	For Windows, there is a helper called \href{http://gitcredentialstore.codeplex.com/}{git-credential-winstore.exe} or \href{https://stackoverflow.com/questions/11693074/git-credential-cache-is-not-a-git-command}{wincred in msysgit}.
\end{itemize}
\section{What if a file is no longer being tracked by git?}
The only way that I found is this (keep a copy of file beforehand.)

\begin{bashcode}
	git rm --cached path/to/file
	git reset path/to/file
\end{bashcode}

\section{Working with remote branches}
\begin{itemize}
	\item Checkout a remote branch: \ic{git co hotfix/my_fix}
	\item Delete a remote branch: \ic{git push origin --delete hotfix/my_fix}
\end{itemize}

\section{Git flow}
See this \href{http://danielkummer.github.io/git-flow-cheatsheet/}{document}.