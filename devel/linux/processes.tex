\section{Linux Processes}
\subsection{Getting ProcessInfo}
\begin{lstlisting}[language=bash,caption={Getting pid}]
pidof httpd
pidof apache2
pidof firefox
\end{lstlisting}

\begin{lstlisting}[language=bash,caption={Getting all pids}]
ps aux | grep httpd
ps aux | grep apache2
ps aux | grep  firefox

pgrep firefox
# list the process called sshd which is owned root
pgrep -u root sshd
\end{lstlisting}

\begin{lstlisting}[language=bash,caption={Process Tree}]
# To display a tree of processes
pstree
# Print a process tree using ps
ps -ejH
ps axjf
\end{lstlisting}

\begin{lstlisting}[language=bash,caption={Get SecurityInfo}]
ps -eo euser,ruser,suser,fuser,f,comm,label
ps axZ
ps -eM
\end{lstlisting}

\subsection{Killing processes}
\begin{lstlisting}[language=bash,caption={Kill command}]
kill [signal] PID
kill -15 PID
kill -9 PID
kill -SIGTERM PID
kill [options] -SIGTERM PID
kill -9  pid1 pid2 pid3
\end{lstlisting}

\begin{lstlisting}[language=bash,caption={KillAll command}]
killall Process-Name-Here
killall -15 lighttpd
killall -9 lighttpd
killall -9 firefox-bin
\end{lstlisting}
\subsection{htop}
\textbf{htop} is interactive process viewer just like top, but allows to scroll the list vertically and horizontally to see all processes and their full command lines. Tasks related to processes (killing, renicing) can be done without entering their PIDs.
\begin{lstlisting}[language=bash,caption={installing htop}]
sudo apt-get install htop
yum install htop
htop
\end{lstlisting}
\subsection{atop}
The program \textbf{atop} is an interactive monitor to view the load on a Linux system. It shows the occupation of the most critical hardware resources (from a performance point of view) on system level, i.e. cpu, memory, disk and network. It also shows which processes are responsible for the indicated load with respect to cpu- and memory load on process level;

\begin{lstlisting}[language=bash,caption={atop}]
atop
\end{lstlisting}
