\section{Java}
\subsection{EE Features}
\subsubsection{Embeddable}
\begin{javacode}[Embeddable]
	@Entity 
	public class ReportCostEntity implements  Serializable {
		
		@Id
		private Long id;
		
		@Embedded   
		@AttributeOverrides( {
			@AttributeOverride(name="coveredByGrant.amount", column = @Column(name="contracted_coveredByGrant") ),
			@AttributeOverride(name="foundedFromOwnResources.amount", column = @Column(name="contracted_foundedFromOwnResources")),
			@AttributeOverride(name="personalContribution.amount", column = @Column(name="contracted_personalContribution"))
		} )
		private ReportCostValues contracted;
		
		@Embedded
		@AttributeOverrides( {
			@AttributeOverride(name="coveredByGrant.amount", column = @Column(name="current_coveredByGrant") ),
			@AttributeOverride(name="foundedFromOwnResources.amount", column = @Column(name="current_foundedFromOwnResources")),
			@AttributeOverride(name="personalContribution.amount", column = @Column(name="current_personalContribution"))
		} )
		private ReportCostValues current;
		
		@Embedded 
		@AttributeOverrides( {
			@AttributeOverride(name="coveredByGrant.amount", column = @Column(name="previousReport_coveredByGrant") ),
			@AttributeOverride(name="foundedFromOwnResources.amount", column = @Column(name="previousReport_foundedFromOwnResources")),
			@AttributeOverride(name="personalContribution.amount", column = @Column(name="previousReport_personalContribution"))
		} )
		private ReportCostValues previousReport;
	} 
\end{javacode}
	

\subsection{Resources}
In order to access a resource we can use:

\begin{javacode}[Accessing a resource through bundle:]
    ResourceBundle messageBundle = ResourceBundle.getBundle("ValidationMessages");
    return messageBundle.getString(messageId);
\end{javacode}

In order that the resource would be available in both Test and the Production we should put them in \ic{src/main/resources}. For example for the above example if we put \ic{ValidationMessages.properties} in the specified folder it would be accessible in both unit tests and production server.
\subsection{String.format}
Samples:

\begin{javacode}
	String formattedString = String.format("Order with OrdId : %d and Amount: %d is missing", 40021, 3000);
	 System.out.printf("Amount : %08d %n" , 221);
	 
	 System.out.printf("positive number : +%d %n", 1534632142);
	 System.out.printf("negative number : -%d %n", 989899);
	 
	 //printing floating point number with System.format()
	 System.out.printf("%f %n", Math.E);
	 
	 //3 digit after decimal point
	 System.out.printf("%.3f %n", Math.E);
	 
	 //8 charcter in width and 3 digit after decimal point
	 System.out.printf("%8.3f %n", Math.E);
	 
	 //adding comma into long numbers
	 System.out.printf("Total %,d messages processed today", 10000000);
\end{javacode}
