\section{Java}

\begin{javacode}[Embeddable]
	@Entity 
	public class ReportCostEntity implements  Serializable {
		
		@Id
		private Long id;
		
		@Embedded   
		@AttributeOverrides( {
			@AttributeOverride(name="coveredByGrant.amount", column = @Column(name="contracted_coveredByGrant") ),
			@AttributeOverride(name="foundedFromOwnResources.amount", column = @Column(name="contracted_foundedFromOwnResources")),
			@AttributeOverride(name="personalContribution.amount", column = @Column(name="contracted_personalContribution"))
		} )
		private ReportCostValues contracted;
		
		@Embedded
		@AttributeOverrides( {
			@AttributeOverride(name="coveredByGrant.amount", column = @Column(name="current_coveredByGrant") ),
			@AttributeOverride(name="foundedFromOwnResources.amount", column = @Column(name="current_foundedFromOwnResources")),
			@AttributeOverride(name="personalContribution.amount", column = @Column(name="current_personalContribution"))
		} )
		private ReportCostValues current;
		
		@Embedded 
		@AttributeOverrides( {
			@AttributeOverride(name="coveredByGrant.amount", column = @Column(name="previousReport_coveredByGrant") ),
			@AttributeOverride(name="foundedFromOwnResources.amount", column = @Column(name="previousReport_foundedFromOwnResources")),
			@AttributeOverride(name="personalContribution.amount", column = @Column(name="previousReport_personalContribution"))
		} )
		private ReportCostValues previousReport;
	} 
\end{javacode}

\paragraph{How maven works?}
\begin{itemize}
	\item There are different settings that should be modified locally in-order for goals to work. For example the following items should be added to \ic{\~/.m2/settings.xml} file: \newline
	\begin{javacode}[profile]
	    <profiles>
		    <profile>
			    <id>glassfish-profile</id>
			    <properties>
				    <glassfish.user>admin</glassfish.user>
				    <glassfish.password>nassim</glassfish.password>
				    <glassfish.domain.host>gammer.se.company.com</glassfish.domain.host>
				    <glassfish.local.directory>/Users/rezasa/glassfish4</glassfish.local.directory>
				    <glassfish.domain.name>domain1</glassfish.domain.name>
				    <glassfish.domain.httpPort>8080</glassfish.domain.httpPort>
				    <glassfish.domain.adminPort>4848</glassfish.domain.adminPort>
				    <migrate_db_url>jdbc:mysql://gammer.se.company.com:3306/crm_test</migrate_db_url>
				    <migrate_db_user>pos_test</migrate_db_user>
				    <migrate_db_user_password>pos_test</migrate_db_user_password>
			    </properties>
		    </profile>
	    </profiles>
	\end{javacode}
	\item It is important for some commands to be in the correct directory and then execute them. For example in \ic{pos-services}. A list of known commands to me are:
	\begin{itemize}
		\item \ic{mvn clean test}\newline
		Then you go to for example \ic{pos-services/target/surefire-reports} or \ic{pos-services/target/jacoco-ut} to view the reports and see the coverage.
		\item \ic{mvn exec:exec -P glassfish-profile,db-migrate}\newline
		Use this in \ic{pos-services} folder to do database migration
		\item Another possibility is to recompile and redeploying\newline
		\ic{mvn compiler:compile glassfish:redeploy -P glassfish-profile}
	\end{itemize}
\end{itemize}
	

\subsection{Resources}
In order to access a resource we can use:

\begin{javacode}[Accessing a resource through bundle:]
    ResourceBundle messageBundle = ResourceBundle.getBundle("ValidationMessages");
    return messageBundle.getString(messageId);
\end{javacode}

In order that the resource would be available in both Test and the Production we should put them in \ic{src/main/resources}. For example for the above example if we put \ic{ValidationMessages.properties} in the specified folder it would be accessible in both unit tests and production server.
\subsection{Unit Testing}
\subsection{String.format}
Samples:

\begin{javacode}
	String formattedString = String.format("Order with OrdId : %d and Amount: %d is missing", 40021, 3000);
	 System.out.printf("Amount : %08d %n" , 221);
	 
	 System.out.printf("positive number : +%d %n", 1534632142);
	 System.out.printf("negative number : -%d %n", 989899);
	 
	 //printing floating point number with System.format()
	 System.out.printf("%f %n", Math.E);
	 
	 //3 digit after decimal point
	 System.out.printf("%.3f %n", Math.E);
	 
	 //8 charcter in width and 3 digit after decimal point
	 System.out.printf("%8.3f %n", Math.E);
	 
	 //adding comma into long numbers
	 System.out.printf("Total %,d messages processed today", 10000000);
\end{javacode}
