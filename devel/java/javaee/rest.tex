\section{Rest Services}
\subsection{Resource Naming}
\begin{itemize}
	\item \textbf{Pluralization}:\newline
		There are good arguments on both sides (pluralizers and the "singularizers"), but the commonly-accepted practice is to always use plurals in node names to keep your API URIs consistent across all HTTP methods. The reasoning is based on the concept that customers are a collection within the service suite and the ID (e.g. 33245) refers to one of those customers.
		Using this rule, an example multi-node URI using pluralization would look like (emphasis added):
		
		GET \url{http://www.example.com/customers/33245/orders/8769/lineitems/1}

		However, there are cases that it doesn't make sense. like:\newline
		GET|PUT|DELETE \url{http://www.example.com/configuration}
	\item \textbf{The official rules for URIs:}\newline
		See Internet Engineering Task Force (IETF) - \href{http://tools.ietf.org/html/rfc6570}{URI-Template}

\end{itemize}

\subsection{Wildfly-Rest Skeleton}
In order for rest end points be discoverable you should have a \ic{src/main/webapp/WEB-inf/web.xml} containing the following xml. The code comes from this 
\href{http://docs.jboss.org/resteasy/docs/2.2.1.GA/userguide/html/Installation_Configuration.html}{page}.

\begin{xmlcode}
<web-app>
    <display-name>Archetype Created Web Application</display-name>

    <servlet>
        <servlet-name>RezaTest</servlet-name>

        <servlet-class>
            org.jboss.resteasy.plugins.server.servlet.HttpServletDispatcher
        </servlet-class>

        <init-param>
            <param-name>javax.ws.rs.Application</param-name>
            <param-value>com.existme.testee.MyApp</param-value>
        </init-param>
    </servlet>

    <servlet-mapping>
        <servlet-name>RezaTest</servlet-name>
        <url-pattern>/*</url-pattern>
    </servlet-mapping>
</web-app>
\end{xmlcode}

Or it can be shortened into this form:

\begin{xmlcode}
<web-app xmlns:xsi="http://www.w3.org/2001/XMLSchema-instance"
        xmlns="http://java.sun.com/xml/ns/javaee"
        xsi:schemaLocation="http://java.sun.com/xml/ns/javaee http://java.sun.com/xml/ns/javaee/web-app_3_0.xsd"
        id="WebApp_ID" version="3.0">
    <servlet-mapping>
        <servlet-name>javax.ws.rs.core.Application</servlet-name>
        <url-pattern>/*</url-pattern>
    </servlet-mapping>
</web-app>
\end{xmlcode}