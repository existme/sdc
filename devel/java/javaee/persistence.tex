\section{Persistence}
\subsection{PersistenceProvider}
In order to PersistenceProvider work in Intellij the following dependency should be added to pom.xml and maven depenedencies should be re imported in InteliJ.
The JQL console should be closed and reopened for settings to be applied.

\begin{xmlcode}
	<dependency>
	<groupId>org.eclipse.persistence</groupId>
	<artifactId>eclipselink</artifactId>
	<version>2.6.2</version>
	</dependency>
	
\end{xmlcode}
\subsection{Persistence Unit}
\ic{persistence.xml} should reside in \ic{src/main/resources/META_INF/persistence.xml}.

A sample persistence unit for eclipselink looks like this:

\begin{xmlcode}
<?xml version="1.0" encoding="UTF-8"?>
<persistence xmlns="http://java.sun.com/xml/ns/persistence" version="2.0">
	<persistence-unit name="pu" transaction-type="JTA">
		<provider>org.eclipse.persistence.jpa.PersistenceProvider</provider>
		<class>com.existme.testee.CityEntity</class>
		<jta-data-source>java:/rezasql</jta-data-source>
		<exclude-unlisted-classes>false</exclude-unlisted-classes>		
		<properties>
			<property name="eclipselink.jdbc.url" value="jdbc:mysql://localhost:3306/world"/>
			<property name="eclipselink.jdbc.driver" value="com.mysql.jdbc.Driver"/>
			<property name="eclipselink.jdbc.user" value="root"/>
			<property name="eclipselink.jdbc.password" value="nassim"/>
		</properties>
	</persistence-unit>
</persistence>
\end{xmlcode}

\textbf{**} <jta-data-source> above is important to get rid of jta error.

\textbf{**Setup eclipselinks for wildfly}
Follow the instructions at this \href{http://giordanomaestro.blogspot.se/2015/02/install-jdbc-driver-on-wildfly.html}{page}. Latest version of eclipse.jar could be find \href{http://central.maven.org/maven2/mysql/mysql-connector-java/}{here}.
\textbf{**} Sometimes the injection of PersistenceUnit is not possible and you can use PersistenceContext instead.
\vspace{1cm}
Inject persistence unit using:

\begin{javacode}
@PersistenceContext(unitName = "pu")
EntityManager em;
\end{javacode}

Query persistence unit using:

\begin{javacode}[createQuery sample]
@GET
@Path("/city")
@Produces("application/json")
public Response listCities() {
   	List resultList = em.createQuery("select c from CityEntity c").getResultList();
   	return Response.status(200).entity(resultList).build();
}
\end{javacode}
