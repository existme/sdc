\section{Arquillian}

\subsection{Multiple `Deployments`}
You can have multiple deployments by having several \ic{@Deployment(name="DEPLOYMENT_NAME")} annotations.
Later you can annotate test cases by \ic{@OperateOnDeployment} to indicate which test cases run with which deployment archive.
You can also use \ic{@TargetsContainer} on a deployment to run that specific deployment on a specific container.

\subsection{Container Testing}

Types of different containers:
\begin{itemize}
	\item \textbf{Embedded containers} : Run on the same JVM as your tests.
	\item \textbf{Managed containers} : Arquillian will start during the tests and shuts down after the tests are run.
	\item \textbf{Remote containers} : Which are assumed to be running.
\end{itemize}

\subsubsection{Deployments}
Arquillian uses a static method defined in your test case to create the archive. The
method signature roughly is \ic{public static Archive<?> createTestArchive()}.
This method must be annotated \ic{@Deployment} – this will tell Arquillian to find
the archive and execute it to create the test content. Each of these archive types
also supports different kinds of content.
\subsubsection{Protocols}
There are 4 major protocols for arquillian:
\begin{itemize}
	\item \textbf{Local}
	\item \textbf{Servlet 2.5}
	\item \textbf{Servlet 3.0}
	\item \textbf{JMX}
\end{itemize}

\subsubsection{arquillian.xml}
This file is located in \ic{src/test/resources}. The base skeleton is like this:

\begin{xmlcode}[arquillian.xml]
<arquillian xmlns="http://jboss.org/schema/arquillian"
            xmlns:xsi="http://www.w3.org/2001/XMLSchema-instance"
            xsi:schemaLocation="
        http://jboss.org/schema/arquillian
        http://jboss.org/schema/arquillian/arquillian_1_0.xsd">
    <container qualifier="(*@\hl{reza-wildfly}@*)" default="true">
        <configuration>
            <property name="jbossHome">C:\unix\wildfly-8.1.0.Final</property>
        </configuration>
    </container>
</arquillian>
\end{xmlcode}

The container "reza-wildfly" should be activated in POM file so in POM file we should have a section to activate it.

\begin{xmlcode}[wildfly81-embedded profile in POM.xml]
<profile>
    <id>wildfy81-embedded</id>
    <!-- the dependencies for Wildfly8.1  -->
    <dependencies>
        <dependency>
            <groupId>org.wildfly</groupId>
            <artifactId>wildfly-arquillian-container-embedded</artifactId>
            <version>8.1.0.Final</version>
        </dependency>
        <dependency>
            <groupId>org.wildfly</groupId>
            <artifactId>wildfly-embedded</artifactId>
            <version>8.1.0.Final</version>
        </dependency>
    </dependencies>

    <build>
        <plugins>
            <plugin>
                <groupId>org.apache.maven.plugins</groupId>
                <artifactId>maven-dependency-plugin</artifactId>
                <version>2.8</version>
            </plugin>
            <plugin>
                <groupId>org.apache.maven.plugins</groupId>
                <artifactId>maven-surefire-plugin</artifactId>
                <version>2.17</version>
                <configuration>
                    <systemPropertyVariables>
                        <java.util.logging.manager>org.jboss.logmanager.LogManager</java.util.logging.manager>
                        <jboss.home>${wildflyhome}</jboss.home>
                        <module.path>${wildflyhome}/modules</module.path>
                        <arquillian.launch>(*@\hl{reza-wildfly}@*)</arquillian.launch>
                    </systemPropertyVariables>
                    <redirectTestOutputToFile>false</redirectTestOutputToFile>
                </configuration>
            </plugin>
        </plugins>
    </build>
</profile>
\end{xmlcode}
