\section{Resource Naming}
\begin{itemize}
	\item \textbf{Pluralization}:\newline
		There are good arguments on both sides (pluralizers and the "singularizers"), but the commonly-accepted practice is to always use plurals in node names to keep your API URIs consistent across all HTTP methods. The reasoning is based on the concept that customers are a collection within the service suite and the ID (e.g. 33245) refers to one of those customers.
		Using this rule, an example multi-node URI using pluralization would look like (emphasis added):
		
		GET \url{http://www.example.com/customers/33245/orders/8769/lineitems/1}

		However, there are cases that it doesn't make sense. like:\newline
		GET|PUT|DELETE \url{http://www.example.com/configuration}
	\item \textbf{The official rules for URIs:}\newline
		See Internet Engineering Task Force (IETF) - \href{http://tools.ietf.org/html/rfc6570}{URI-Template}

\end{itemize}