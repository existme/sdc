\section{TexStudio}

\paragraph{How to write scripts for TexStudio?}
\begin{itemize}
	\item see \href{http://texstudio.sourceforge.net/manual/current/usermanual_en.html#SECTION33}{manual}.
\end{itemize}

\paragraph{Writing Macros}
Go to Macros->Edit Macro

\begin{javacode}[]
%SCRIPT
txt = cursor.selectedText()
editor.write("\\ic{"+txt+"}")
cursor.clearSelection()
\end{javacode}

Or you can use the normal with and then use different \ic{\%<defaultcode\%>} as codeblock place holders:

\begin{javacode}[Using Normal Text]
	$\backslash$begin{javacode}[%<title%>]
		%<codeblock%>
	$\backslash$end{javacode}
\end{javacode}

Finally using abbreviations you can easily recall a macro. It's important to use backslash($\backslash$) at the beginning of the abbreviation to let the command be executed.
